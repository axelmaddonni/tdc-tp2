\subsection{Université de Laval}

A continuación presentamos los resultados de realizar una \textit{traceroute} a la página de la Université de Laval\footnote{www2.ulaval.ca.}.

%\begin{table*}[t]
%  \centering
%\begin{tabular}{*{7}{c}}
%    TTL & Gateway       & RTT   & Rel  RTT & ZRTT          & Lugar         & Es ruta submarina? \\
%\hline
%     1 &   192.168.1.1   &  0.723  &  0.723 & -0.455  &               *                  &--- \\       
%     2 &  190.195.130.1  &  73.989 & 73.266 &  1.484  &    Buenos Aires, Argentina       &--- \\       
%     3 &        *        &    *    &    *    &    *     &               *                  &--- \\       
%     4 &        *        &    *    &    *    &    *     &               *                  &--- \\       
%     5 &        *        &    *    &    *    &    *     &               *                  &--- \\       
%     6 &  200.89.161.93  &  30.142 &   0.0  & -0.475  &  Villa Del Parque, Argentina     &--- \\       
%     7 &   200.89.165.1  &  26.662 &   0.0  & -0.475  &  Villa Del Parque, Argentina     &--- \\       
%     8 &  200.89.165.250 &  25.699 &   0.0  & -0.475  &  Villa Del Parque, Argentina     &--- \\       
%     9 &  190.216.88.33  &  21.446 &   0.0  & -0.475  &    Buenos Aires, Argentina       &--- \\       
%    10 &   67.17.94.249  & 155.762 & 134.315&  3.116  &         United States            &Sí  \\  
%    11 &        *        &    *    &    *    &    *     &               *                  &--- \\       
%    12 &    4.69.141.5   & 218.498 & 62.736 &  1.202  &         United States            &--- \\       
%    13 &    4.15.16.6    &  198.75 &   0.0  & -0.475  &  New Brunswick, United States    &--- \\       
%    14 &   192.77.63.70  & 180.916 &   0.0  & -0.475  &       Montréal, Canada           &--- \\      
%    15 &        *        &    *    &    *    &    *     &               *                  &--- \\       
%    16 &        *        &    *    &    *    &    *     &               *                  &--- \\       
%    17 & 206.167.128.158 & 189.741 &  8.825 & -0.239  &       Montréal, Canada           &--- \\      
%    18 & 132.203.244.155 & 189.867 &  0.126 & -0.471  &        Québec, Canada            &--- \\      
%    19 & 132.203.244.131 & 189.704 &   0.0  & -0.475  &        Québec, Canada            &--- \\      
%    20 &  132.203.253.48 & 193.871 &  4.168 & -0.363  &        Québec, Canada            &--- \\      
%    21 & 132.203.252.193 & 191.233 &   0.0  & -0.475  &        Québec, Canada            &--- \\      
%    22 & 132.203.187.106 & 190.367 &   0.0  & -0.475  &        Québec, Canada            &--- \\      
%\end{tabular}
%    \caption{\normalfont \textit{Traceroute} a www2.ulaval.ca.}
%\end{table*}

\begin{figure}[H]
    \centering
    \includegraphics[width=8.5cm]{img/grafico1-www2-ulaval-ca.pdf}
    \caption{\normalfont RTTs entre saltos. El valor asignado al $i$-ésimo nodo corresponde al salto entre el $i$-ésimo y el $i - 1$-ésimo nodo. Para el primer nodo se utiliza simplemente su RTT.}
    \label{graf1L}
\end{figure}

\par En la realización de la \textit{traceroute}, 6 de los 22 ($27.\overline{27}\%$) nodos no respondieron los \textit{Time exceeded}.

\par Observamos una cantidad de \textit{outliers} y una desviación consistente entre los distintos saltos en la figura \ref{graf1L}.
Esto posiblemente se debe a una congestión momentárea en la red durante ciertas repticiones de la \textit{traceroute}, resultanto en que todos los RTTs de la repetición sean considerados \textit{outliers}.

\par Si bien la mayoría de los nodos presentan desviaciones similares, en el de dirección 4.69.141.5 se observa una varianza excepcionalmente alta.
Esto se puede deber a algunos de los factores mencionados en Jobst 2012\cite{anomalias}: métodos de balanceo de carga variando las rutas de los paquetes de diversas mediciones, o caminos de ida/vuelta asimétricos resultando en RTTs inconsistentes.

\begin{figure}[H]
    \centering
    \includegraphics[width=8.5cm]{img/grafico2-www2-ulaval-ca.pdf}
    \caption{\normalfont ZRTTs entre saltos.}
\end{figure}

\begin{figure}[H]
    \centering
    \includegraphics[width=8.5cm]{img/grafico3-www2-ulaval-ca.pdf}
    \caption{\normalfont Ubicación geográfica estimada de la ruta tomada.}
    \label{mapaLaval}
\end{figure}

\par Sólo un salto (de 200.89.165.250 a 67.17.94.249) se presentó como \textit{outlier} según el método planteado\cite{outliers}.
De acuerdo a la estimación geográfica, este salto se da entre Buenos Aires y Estados Unidos.

\par No hay ningún cable submarino que una directamente a Argentina con Estados Unidos.
Sin embargo se pueden trazar rutas submarinas componiendo distintos cables submarinos entre ambos extremos\footnote{Por ejemplo, el South American Crossing (SAC) une a Argentina con Panamá y el Pan-American Crossing (PAC) a Panamá con Estados Unidos.}.

\par Si bien estos dos hechos parecen contradecirse, Jobst 2012\cite{anomalias} provee una explicación: routers de \textit{MPLS}\footnote{\textit{Multiprotocol Label Switching (MPLS)} es un protocolo empleado para simplificar el \textit{forwarding}.} que no honran el campo \textit{TTL}, y por ende no aparecen en la \textit{traceroute}.
Los cables submarinos manejan un flujo muy alto de paquetes, por lo que es esperable que busquen optimizar la performance; ignorar el campo \textit{TTL} es una forma de logarlo.
Esto resulta en que toda la ruta entre Buenos Aires y Estados Unidos se vea condensada en un único salto.

\par Cabe destacar que la herramienta geógrafica utilizada no pudo proveer la latitud y longitud de la dirección IP 67.17.94.249, por lo que el mapa emplea la próxima dirección ubicable, la 4.69.141.5.
