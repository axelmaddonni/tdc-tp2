\section{Introducci\'on te\'orica}

En este trabajo nos proponemos usar un traceroute junto con herramientas estadísticas para intentar detectar enlaces submarinos en las rutas IP. 

Nuestra implementación de traceroute se basará en paquetes ICMP de tipo \texttt{echo request}. Enviaremos, al destino especificado, paquetes ICMP echo request con distintos TTL, esperando que el paquete se quede sin tiempo en un nodo intermedio, y este nodo intermedio nos mande otro paquete ICMP (esta vez de tipo \texttt{time exceeded}) y de esta manera sabremos la dirección IP del nodo intermedio.


Más adelante introduciremos más en detalle cómo fue hecha la implementación de traceroute. Sin embargo, lo que más nos interesa del traceroute es el RTT entre cada nodo: como podemos tomar el tiempo que tarda un paquete en ir y volver de cada nodo intermedio de la ruta, podemos estimar el tiempo que tarda el salto entre cada nodo de la ruta. De esta manera, como nuestro objetivo es detectar enlaces submarinos, podemos pensar que una buena forma de detectarlos sería detectar cuando el salto tarda mucho.

Esto es efectivamente lo que haremos, tendremos una variable aleatoria que será el RTT relativo, es decir, entre saltos, y tomaremos todos los RTT relativos de nuestro traceroute como muestras de esa variable aleatoria. Luego, usando algún modelo de detección de outliers, intentaremos detectar mediciones que corresponden con enlaces submarinos.

Como método de detección de outliers utilizaremos el método descripto en \cite{outliers}. Este método se basa en normalizar la variable aleatoria, para obtener una nueva variable aleatoria que llamaremos ZRTT. Es decir, supondremos que RTT rel (los RTT relativos) son una $N(\mu, \sigma)$ para algún $\mu$, $\sigma$ y normalizaremos para obtener una $N(0,1)$.

\[
  ZRTT = \frac{RTT_i - \overline{RTT}}{s_{RTT}}
\]

Notar que sin embargo este método es imperfecto, porque la variable aleatoria podría no ser (y de hecho no es) una normal.

Ahora bien, tenemos que elegir algún valor límite tal que si $ZRTT$ es mayor que ese valor, diremos que es un outlier, y en consecuencia ese enlace es un enlace submarino. Aquí, es donde el paper \cite{outliers} propone utilizar la variable $\tau$ de Thompson. La $\tau$ de Thompson se calcula como sigue

\[
  \tau = \frac{
                 t_{\frac{\alpha}{2}} (n-1)
              }{
                 \sqrt{n} \sqrt{n-2 + t_{\frac{\alpha}{2}}^2}
              }
\]

Donde $n$ es la cantidad de mediciones. 

Todo esto resume la parte teórica del problema. De aquí en adelante nos adentraremos en la implementación de la herramienta, y en la experimentación.




