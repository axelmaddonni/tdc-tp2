\section{Introducci\'on te\'orica}

En este trabajo nos proponemos usar un traceroute junto con herramientas estadísticas para intentar detectar enlaces submarinos en las rutas IP. 

Nuestra implementación de traceroute se basará en paquetes ICMP de tipo \texttt{echo request}. Enviaremos, al destino especificado, paquetes ICMP echo request con distintos TTL, esperando que el paquete se quede sin tiempo en un nodo intermedio, y este nodo intermedio nos mande otro paquete ICMP (esta vez de tipo \texttt{time exceeded}) y de esta manera sabremos la dirección IP del nodo intermedio.


Más adelante introduciremos más en detalle cómo fue hecha la implementación de traceroute. Sin embargo, lo que más nos interesa del traceroute es el RTT entre cada nodo: como podemos tomar el tiempo que tarda un paquete en ir y volver de cada nodo intermedio de la ruta, 
