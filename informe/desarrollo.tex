\section{Desarrollo}

\PARstart Como explicamos anteriormente, \ldots


\subsection{Primera consigna}

\subsection{Segunda consigna}

\par Cabe destacar que en el cálculo de los ZRTTs, sólo tomamos valores no-negativos\footnote{Consideramos como 0 a los valores negativos.} de los RTTs entre saltos. 
Sin embargo, en los gráficos de RTTs que presentamos, los valores pueden ser menores a 0.
De forma contraria las figuras proveen información confusa y difícil de interpretar.

\subsection{Conceptos generales}
