\section{Conclusión}

\PARstart Observamos algunos de los comportamientos anómalos mencionados en Jobs 2012\cite{anomalias}: el más ubicuo es el primero descripto en el paper, donde ciertos nodos no responden los \textit{Time exceeded}. 
También observamos el caso donde routers \textit{MPLS} no aparecen en la \textit{traceroute}.

\par En el experimento a la Universidad de Laval se observa una gran cantidad de RTTs entre saltos negativos.
Esto puede ser un ejemplo de la anomalía "False Roundtrip Times" mencionada en el paper: la ruta de vuelta es asimétrica, por lo que los paquetes de \textit{Time exceeded} son forwardeados hacia delante en la ruta antes de volver al origen, resultando en RTTs relativos negativos.

\par Una anomalía que se observa frecuentemente al realizar \textit{traceroutes} pero que no se presentó en ninguno de los experimentos es la de \textit{missing destination}.
El hecho de que no se haya incurrido en ésta puede deberse a una mera coincidencia, o puede ser una consecuencia del tipo de destinos elegidos: por la consigna, todos son universidades.
La típica causa de la anomalía es un \textit{firewall} bloqueando los paquetes \textit{ICMP} al destino; es posible que este \textit{setup} sea preferido por los servers comerciales e ignorado por las instituciones académicas, quizás debido a un mayor enfoque en la \textit{performance} por parte de los primeros.

\par Frecuentemente la herramienta geográfica utilizada realizó estimaciones incorrectas en nodos cercanos a y coincidentes con cables submarinos.
Esta anomalía no es mencionada en el paper ya que no se enfocan en el aspecto geográfico de las \textit{traceroutes}.

\par Si las estimaciones se realizan a partir de ubicaciones iniciales conocidas, y de un análisis estadístico de las ubicaciones conocidas y estimados de los vecinos de los nodos desconocidos, es de esperar que los errores se concentren en las conexiones entre regiones distintas, como en el caso de los cables submarinos.

\par Ningún experimento presentó falsos negativos, y sólo uno resultó en un falso negativo.
Si bien era una de las rutas más largas, no consideramos que éste sea el factor fundamental en la predicción fallida; el experimento a la Universidad de \={O}saka (a una distancia similar) fue satisfactorio.
En el caso del experimento a la Universidad de Alejandría, la ruta submarina mal clasificada era la más corta de todas las analizadas; la longitud de la ruta es un factor mucho menos influyente que su topología: una ruta consistente de pequeños saltos resultará en un $\tau$ mucho menor que una de igual longitud pero con una conexión submarina cubriendo gran parte de su extensión.

\par En el único experimento donde el modelo falló, un cálculo distinto del $\tau$ hubiera sido irrelevante ya que el ZRTT de la conexión marítima mal clasificada es negativo; el falso negativo no podría subsanarse sin un extremo número de falsos positivos.
